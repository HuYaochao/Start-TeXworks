% 注释
\documentclass[UTF8]{ctexart} %中英文排版类,或其他文档类,如 report, book ,article等
\usepackage{amsmath} %数学包
\usepackage{graphicx} %引入图片

%注意,文件(包括 .tex, .jpg 等)用英文,否则易报错

\usepackage{geometry} % 设置页边距
\geometry{papersize={20cm,15cm}}
\geometry{left=1cm,right=2cm,top=3cm,bottom=4cm}

\usepackage{fancyhdr} %设置 页眉页脚
\pagestyle{fancy}
\lhead{\author}
\chead{\date}
\rhead{152xxxxxxxx}
\lfoot{}
\cfoot{\thepage}
\rfoot{}
\renewcommand{\headrulewidth}{0.4pt}
\renewcommand{\headwidth}{\textwidth}
\renewcommand{\footrulewidth}{0pt}

\usepackage{setspace}%行间距
\onehalfspacing %将行距设置为字号的 1.5 倍:

\addtolength{\parskip}{.4em} %段间距



 \usepackage{xeCJK}  % 中文字体扩展管理宏包,务必添加!!

\setCJKfamilyfont{ceyy } {站酷仓耳渔阳体-W02.ttf}
\newcommand{  \ceyy } {  \CJKfamily{ceyy } } %创建新字体调用命令

\title{标题}
\author{作者}
\date{\today}    %不用日期把\today,去掉即可,\date{} ,你也可以手动输入日期


\begin{document} %内容开头   , \end{document}结尾,其他地方的内容不会输出

\ceyy  %调用字体



%之上的部分被称为导言区,设置页面大小、页眉页脚样式、章节标题样式等等。

\maketitle %生成标题页
\tableofcontents  %目录




\section{一级标题}
内容 内容 内容 内容  \% %打印百分号,需用\转义
\subsection{二级标题}
内容内容内容
\subsubsection{三级标题}


\paragraph{(第一)段落标题}
这是段落标题对应的内容。这个内容会直接跟在段落标题后面,而不是在标题下方。

\subparagraph{(第二)段落标题}
这是段落二标题对应的内容


\section{数学}

%行内
Einstein 's $E=m_1c^2$.  %_下标,数字5。^上标,数字6

%%^默认只作用于之后的一个字符,如果想对连续的几个字符起作用,请将这些字符用花括号 {} 括起来,例如:

%$$ ... $$ 来插入不编号的行间公式。但是在 LaTeX 中这样做会改变行文的默认行间距,不推荐。

\[ z = r\cdot e^{2\pi i}. \]

%行间,自动居中
\[ E=mc^3. \]

%加序号,equation,行间
\begin{equation}
E=mc^4.
\end{equation}

%不加序号,equation*,行间
\begin{equation*}
E=mc^5.
\end{equation*}


\subsection{根式与分式}

%根式用 \sqrt{·} 来表示,分式用 \frac{·}{·} 来表示(第一个参数为分子,第二个为分母)。
$\sqrt{x}$, $\frac{1}{2}$.


%可以发现,在行间公式和行内公式中,分式的输出效果是有差异的。如果要强制行内模式的分式显示为行间模式的大小,可以使用 \dfrac, 反之可以使用 \tfrac。
$\sqrt{x}$, $\tfrac{1}{2}$.

\[ \tfrac{1}{2}. \]


\subsection{运算符}

+-*/
\[ \pm\; \times \; \div\; \cdot\; \cap\; \cup\;
\geq\; \leq\; \neq\; \approx \; \equiv \]

%连加、连乘、极限、积分等大型运算符分别用 \sum, \prod, \lim, \int 生成。他们的上下标在行内公式中被压缩,以适应行高。我们可以用 \limits 和 \nolimits 来强制显式地指定是否压缩这些上下标。例如:

%\quad 是一个水平间距命令,用于在文本中插入一个固定宽度的水平空格。这个空格的宽度大约是当前字体大小的两倍。它通常用于在句子或单词之间创建额外的间距,使得文本更加易读或排版更美观。

$ \sum_{i=1}^n i\quad \prod_{i=1}^n $
$ \sum\limits _{i=1}^n i\quad \prod\limits _{i=1}^n $
\[ \lim_{x\to0}x^2 \quad \int_a^b x^2 dx \]
\[ \lim\nolimits _{x\to0}x^2\quad \int\nolimits_a^b x^2 dx \]

%多重积分可以使用 \iint, \iiint, \iiiint, \idotsint 等命令输入。
\[ \iint\quad \iiint\quad \iiiint\quad \idotsint \]


\subsection{定界符(括号等)}

%各种括号用 (), [], \{\}, \langle\rangle 等命令表示;注意花括号通常用来输入命令和环境的参数,所以在数学公式中它们前面要加 \。因为 LaTeX 中 | 和 \| 的应用过于随意,amsmath 宏包推荐用 \lvert\rvert 和 \lVert\rVert 取而代之。

%为了调整这些定界符的大小,amsmath 宏包推荐使用 \big, \Big, \bigg, \Bigg 等一系列命令放在上述括号前面调整大小。

\[ \Biggl(\biggl(\Bigl(\bigl((x)\bigr)\Bigr)\biggr)\Biggr) \]
\[ \Biggl[\biggl[\Bigl[\bigl[[x]\bigr]\Bigr]\biggr]\Biggr] \]
\[ \Biggl \{\biggl \{\Bigl \{\bigl \{\{x\}\bigr \}\Bigr \}\biggr \}\Biggr\} \]
\[ \Biggl\langle\biggl\langle\Bigl\langle\bigl\langle\langle x
\rangle\bigr\rangle\Bigr\rangle\biggr\rangle\Biggr\rangle \]
\[ \Biggl\lvert\biggl\lvert\Bigl\lvert\bigl\lvert\lvert x
\rvert\bigr\rvert\Bigr\rvert\biggr\rvert\Biggr\rvert \]
\[ \Biggl\lVert\biggl\lVert\Bigl\lVert\bigl\lVert\lVert x
\rVert\bigr\rVert\Bigr\rVert\biggr\rVert\Biggr\rVert \]


\subsection{省略号}

%省略号用 \dots, \cdots, \vdots, \ddots 等命令表示。\dots 和 \cdots 的纵向位置不同,前者一般用于有下标的序列。

\[ x_1,x_2,\dots ,x_n\quad 1,2,\cdots ,n\quad
\vdots\quad \ddots \]

\subsection{矩阵}

\[ \begin{pmatrix} a&b\\c&d \end{pmatrix} \quad
\begin{bmatrix} a&b\\c&d \end{bmatrix} \quad
\begin{Bmatrix} a&b\\c&d \end{Bmatrix} \quad
\begin{vmatrix} a&b\\c&d \end{vmatrix} \quad
\begin{Vmatrix} a&b\\c&d \end{Vmatrix} \]

%使用 smallmatrix 环境,可以生成行内公式的小矩阵。
Marry has a little matrix $ ( \begin{smallmatrix} a&b\\c&d \end{smallmatrix} ) $.


\subsection{多行公式}

不对齐
\begin{multline}
x = a+b+c+{} \\
d+e+f+g
\end{multline}
如果不需要编号,可以使用 multline* 环境代替。

对齐

\[\begin{aligned}
x ={}& a+b+c+{} \\
&d+e+f+g
\end{aligned}\]

公式组

无需对齐的公式组可以使用 gather 环境,需要对齐的公式组可以使用 align 环境。他们都带有编号,如果不需要编号可以使用带星花的版本。

\begin{gather}
a = b+c+d \\
x = y+z
\end{gather}
\begin{align}
a &= b+c+d \\
x &= y+z
\end{align}

分段函数

分段函数可以用cases次环境来实现,它必须包含在数学环境之内。

\[ y= \begin{cases}
-x,\quad x\leq 0 \\
x,\quad x>0
\end{cases} \]


\section{插入图片和表格}

同级目录
\includegraphics{a.jpg}

%图片可能很大,超过了输出文件的纸张大小,或者干脆就是你自己觉得输出的效果不爽。这时候你可以用 \includegraphics 控制序列的可选参数来控制。比如

\includegraphics[width = .8\textwidth]{a.jpg}

这样图片的宽度会被缩放至页面宽度的百分之八十,图片的总高度会按比例缩放。

\section{表格}

%tabular 环境提供了最简单的表格功能。它用 \hline 命令表示横线,在列格式中用 | 表示竖线;用 & 来分列,用 \\ 来换行;每列可以采用居左、居中、居右等横向对齐方式,分别用 l、c、r 来表示。

\begin{tabular}{|l|c|r|}
 \hline
操作系统& 发行版& 编辑器\\
 \hline
Windows & MikTeX & TexMakerX \\
 \hline
Unix/Linux & teTeX & Kile \\
 \hline
Mac OS & MacTeX & TeXShop \\
 \hline
通用& TeX Live & TeXworks \\
 \hline
\end{tabular}

浮动体

%插图和表格通常需要占据大块空间,所以在文字处理软件中我们经常需要调整他们的位置。figure 和 table 环境可以自动完成这样的任务;这种自动调整位置的环境称作浮动体(float)。我们以 figure 为例。

%htbp 选项用来指定插图的理想位置,这几个字母分别代表 here, top, bottom, float page,也就是就这里、页顶、页尾、浮动页(专门放浮动体的单独页面或分栏)。\centering 用来使插图居中;\caption 命令设置插图标题,LaTeX 会自动给浮动体的标题加上编号。注意 \label 应该放在标题命令之后。
\begin{figure}[htbp]
\centering
\includegraphics{a.jpg}
\caption{有图有真相}
\label{fig:myphoto}
\end{figure}






\end{document}